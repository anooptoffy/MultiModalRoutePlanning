\documentclass[12pt,a4paper]{article}
\usepackage[utf8]{inputenc}
\usepackage{graphicx}
\usepackage{amsmath}
\usepackage{amsthm}
\usepackage{amssymb}
\usepackage{hyperref}

\begin{document}

\begin{titlepage}
	\centering
	\includegraphics[width=0.15\textwidth]{IIIT-B_logo.jpg}\par\vspace{1cm}
	{\scshape\LARGE International Institute of Information Technology, Bangalore \par}
	\vspace{1cm}
	{\scshape\Large Software Design Document\par}
	{\Large DS 703 Geographic Information Systems\par}
	\vspace{1.5cm}
	{\huge\bfseries Multi Modal Route Planning\par}
	\vspace{2cm}	   
	{\Large\itshape Anoop Toffy - MT2016016\par}
	{\Large\itshape Mehul Singh - MT2016083\par}		 
	{\Large\itshape Harshit Joshi - MT2016059\par}	 
	{\Large\itshape Mohd Awais - MT2016086\par}
	\vfill
	Guide : \par
	Dr. S. Rajagopalan

	\vfill

% Bottom of the page
	{\large \today\par}
\end{titlepage}


\tableofcontents
\listoffigures
\listoftables
\newpage

\section{Problem Statement}
Multi Modal Route Planning is to provide the user with optimal, feasible and personalized route plan between a given source and destination. The system is designed to provide various travel approaches and provides user the flexibility to choose from different modes of transport.

\section{Data Collection and Preprocessing}

The bus and metro schedule is collected from the Bengaluru Metropolitan Transport Corporation and Bangaluru Metro departments. It is then represented as graphs using the spatial data collected using the software QGIS and Transcad. Thus we arrive at the networks with which we can work on and apply heuristics approach to find out the optimal route.

\section{Algorithm for Multi Modal Route Planning}

The proposed algorithm for calculating the Optimal Multi Modal Route is as given below:

\section{Flow Diagrams}

\end{document}